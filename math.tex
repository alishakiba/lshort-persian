%%%%%%%%%%%%%%%%%%%%%%%%%%%%%%%%%%%%%%%%%%%%%%%%%%%%%%%%%%%%%%%%
% Contents: Math typesetting with LaTeX
% $Id: math.tex 169 2008-09-24 07:32:13Z oetiker $
%%%%%%%%%%%%%%%%%%%%%%%%%%%%%%%%%%%%%%%%%%%%%%%%%%%%%%%%%%%%%%%%%
\chapter{حروف‌چینی فرمول‌های ریاضی}
%We beed to switch to the original formula numbers 
\makeatletter
\def\tagform@#1{\maketag@@@{\lr{(\ignorespaces\@@text{#1}\unskip\@@italiccorr)}}}
\makeatother
%At the end of this chapter we will return to xepersian adaption
\begin{intro}

حال آماده هستید! در این فصل به قویترین قسمت تک، حروف‌چینی ریاضی، حمله می‌کنیم. اما توجه داشته باشید، این فصل فقط سطح کار را صیقل می‌دهد. با ‌وجود این که مطالب این فصل برای بسیاری از افراد کافی است، اگر نتوانستید در آن پاسخ بعضی از نیازهای حروف‌چینی ریاضی خود را بیابید نا‌امید نشوید. به احتمال بسیار زیاد جواب شما در  \lr{\AmS-\LaTeX{}} داده شده است.
\end{intro}
\section{\texorpdfstring{کلاف \lr{\texorpdfstring{\AmS}{AMS}-\LaTeX{}}}{کلاف AMS-Latex}}
اگر می‌خواهید حروف‌چینی (پیشرفته)
\wi{ریاضی}\romanindex{mathematics} 
انجام دهید، باید از کلاف 
\lr{\AmS-\LaTeX} 
استفاده کنید. کلاف 
\lr{\AmS-\LaTeX} 
مجموعه‌ای از بسته‌ها و طبقه‌ها برای حروف‌چینی ریاضی است. ما بیشتر به بررسی بستۀ 
\pai{amsmath} 
می‌پردازیم که جزیی از این کلاف است. 
\lr{\AmS-\LaTeX} 
توسط \wi{انجمن ریاضی آمریکا} تولید شده است و به‌طور گسترده برای حروف‌چینی ریاضی مورد استفاده قرار می‌گیرد. خود لاتک دارای محیط‌هایی ابتدایی برای ریاضی است، اما این محیط‌ها محدود هستند 
(یا برعکس: \lr{\AmS-\LaTeX} نامحدود است)
و در بعضی حالات ناپایدار نیز هستند.

\lr{\AmS-\LaTeX} 
جزیی از توزیع مورد نیاز است و توسط تمام توزیع‌های اخیر لاتک ارائه می‌شود.% 
\footnote{اگر آن را ندارید، به 
  \texttt{CTAN:macros/latex/required/amslatex} مراجعه کنید.} 
در این فصل فرض بر این است که \pai{amsmath} در سرآغاز نوشتار‌ فراخوانی شده است:

\begin{code}
\verb|\usepackage{amsmath}|
\end{code}
\section{فرمول‌های تنها}

دو راه برای چیدن یک \wi{فرمول}\romanindex{formulae} وجود دارد: در متن داخل یک پاراگراف 
(\wi{سبک‌ متنی}\Footnote{text style}\romanindex{textstyle})، 
یا پاراگراف می‌تواند برای نمایش جداگانه شکسته شود 
(\wi{سبک‌ نمایشی}\Footnote{display style}\romanindex{display style}). 
فرمول‌های ریاضی {\femph درون} 
متن \romanindex{equation} یک پاراگراف در میان دو نماد  \texttt{\$} وارد می‌شوند:

\begin{example}
Add $a$ squared and $b$ squared
to get $c$ squared. Or, using 
a more mathematical approach:
$a^2 + b^2 = c^2$
\end{example}
\begin{example}
\TeX{} is pronounced as 
$\tau\epsilon\chi$\\[5pt]
100~m$^{3}$ of water\\[5pt]
This comes from my $\heartsuit$
\end{example}

اگر می‌خواهید فرمول‌های بیشتری را جدا از بقیه پاراگراف بنویسید، مناسب‌تر است که آن را \femph{نمایش}
دهید به‌جای آنکه پاراگراف را بشکنید. برای انجام این کار از محیط فرمول استفاده کنید و فرمول‌ها را بین  \verb|\begin{equation}| و
\verb|\end{equation}| قرار دهید.%
\footnote{این یک فرمان \textsf{amsmath} است. اگر به این بسته دسترسی ندارید از محیط \ei{displaymath} مربوط به خود لاتک استفاده کنید.} 
آنگاه می‌توانید به فرمول یک برچسب (\ci{label}) بدهید و در دیگر نقاط نوشتار‌ با فرمان  \ci{eqref} به آن ارجاع دهید. اگر می‌خواهید به فرمول اسم ویژه‌ای بدهید به‌جای این‌کار از فرمان \ci{tag} استفاده کنید. از \ci{eqref} نمی‌توانید برای \ci{tag} استفاده کنید.

\begin{example}
Add $a$ squared and $b$ squared
to get $c$ squared. Or, using
a more mathematical approach
 \begin{equation}
   a^2 + b^2 = c^2
 \end{equation}
Einstein says
 \begin{equation}
   E = mc^2 \label{clever}
 \end{equation}
He didn't say
 \begin{equation}
  1 + 1 = 3 \tag{dumb}
 \end{equation}
This is a reference to 
\eqref{clever}. 
\end{example}


اگر نمی‌خواهید لاتک فرمول‌ها را شماره‌گذاری کند، از شکل ستاره‌دار محیط  \texttt{equation} استفاده کنید، \ei{equation*}، یا حتی آسان‌تر، فرمول را بین دو علامت  \ci{[} و \ci{]} قرار دهید:\footnote{\romanindex{\textsf{amsmath} equation}
  \romanindex{\LaTeX equation} این فرمان دوباره از \textsf{amsmath} است. اگر این بسته را فراخوانی نکرده‌اید، از محیط \texttt{equation} مربوط به خود لاتک استفاده کنید. نام فرمان‌های \lr{\textsf{amsmath}/\LaTeX{}} ممکن است به نظر برسد که کمی گیج‌کننده  هستند، ولی این واقعاً یک مشکل برای کسانی که از این بسته استفاده می‌کنند نیست. بهتر است این بسته را از ابتدا فراخوانی کنید زیرا ممکن است بعداً مجبور به استفاده از آن شوید، و آنگاه محیط‌های غیر‌ شماره‌گذاری شده خود لاتک ممکن است توسط این بسته شماره‌گذاری شود.}
\begin{example}
Add $a$ squared and $b$ squared
to get $c$ squared. Or, using
a more mathematical approach
 \begin{equation*}
   a^2 + b^2 = c^2
 \end{equation*}
or you can type less for the
same effect:
 \[ a^2 + b^2 = c^2 \]
\end{example}

به تفاوت ‌ حروف‌چینی بین \wi{سبک‌ متنی}
 و \wi{سبک‌ نمایشی}
 توجه کنید: 
\begin{example}
This is text style: 
$\lim_{n \to \infty} 
 \sum_{k=1}^n \frac{1}{k^2} 
 = \frac{\pi^2}{6}$.
And this is display style:
 \begin{equation}
  \lim_{n \to \infty} 
  \sum_{k=1}^n \frac{1}{k^2} 
  = \frac{\pi^2}{6}
 \end{equation}
\end{example}

در سبک‌ متنی، عبارات طولانی یا عمیق را در  \ci{smash} محصور کنید. این کار لاتک را وادار می‌سازد ارتفاع عبارت را نادیده بگیرد و باعث یکنواخت شدن فاصله بین خط‌ها می‌شود.

\begin{example}
A $d_{e_{e_p}}$ mathematical
expression  followed by a
$h^{i^{g^h}}$ expression. As
opposed to a smashed 
\smash{$d_{e_{e_p}}$} expression 
followed by a
\smash{$h^{i^{g^h}}$} expression.
\end{example}
\subsection{سبک ریاضی}

همچنین تفاوت‌هایی بین \femph{\wi{سبک ریاضی}}
 و \femph{سبک متنی}
وجود دارد. به عنوان مثال در \femph{سبک ریاضی}:

\begin{enumerate}

\item \romanindex{math mode spacing}\index{فاصله‌گذاری!سبک ریاضی}
بسیاری از فاصله‌ها و شکست خط‌ها در سبک ریاضی بی‌اهمیت هستند، زیرا تمام فاصله‌ها در عبارات ریاضی یا به طور منطقی ایجاد می‌شوند، و یا این که باید توسط فرمان‌هایی مانند  \ci{,} و \ci{quad} یا
\ci{qquad} تولید ‌شوند 
( بعداً به این فرمان‌ها می‌رسیم، بخش 
\ref{sec:math-spacing}
%\LRE{\hyperref[sec:math-spacing]{5.4}} 
 را ببینید).
 
\item خط‌های خالی مجاز نیستند. هر فرمول تنها در یک پاراگراف قرار داده می‌شود.

\item هر حرف به عنوان نام یک متغیر درنظر گرفته می‌شود و به همین منظور چیده می‌شود. اگر می‌خواهید در یک فرمول متن عادی بنویسید (قلم نرمال ایستاده و فاصله نرمال)
آنگاه باید متن را بوسیله فرمان  \verb|\text{...}| وارد کنید 
(همچنین بخش  
\ref{sec:fontsz}
%\LRE{\hyperref[sec:fontsz]{6.4}} 
در صفحه  
\pageref{sec:fontsz} را  ببینید).
\end{enumerate}
\makeatletter\def\text#1{\@@text{#1}}\makeatother%We need to remove some damage produces by farsixetex.tex
\begin{example}
$\forall x \in \mathbf{R}:
 \qquad x^{2} \geq 0$
\end{example}
\begin{example}
$x^{2} \geq 0\qquad
 \text{for all }x\in\mathbf{R}$
\end{example}

ریاضیدان‌ها از نمادهای پیچیده‌ای استفاده می‌کنند: مناسب است که در اینجا از قلـــــم \wi{\lr{blackboard bold}} استفاده کنیم، 
\romanindex{bold symbols} که با استفاده از \ci{mathbb} از بسته  \pai{amssymb} بدست می‌آید.\footnote{\pai{amssymb} قسمتی از کلاف \lr{} نیست، اما ممکن است هنوز قسمتی از توزیع لاتک شما باشد. توزیع خود را بررسی کنید یا به  \texttt{CTAN:/fonts/amsfonts/latex/} بروید و آن را دریافت کنید.}
\ifx\mathbb\undefined\else
آخرین مثال عبارت است از

\begin{example}
$x^{2} \geq 0\qquad
 \text{for all } x 
 \in \mathbb{R}$
\end{example}
\fi

جدول 
\ref{mathalpha}
%\hyperref[mathalpha]{14.4}
 در صفحه 
\pageref{mathalpha}
و جدول  
\ref{mathfonts}
%\LRE{\hyperref[mathfonts]{4.6}}
در صفحه
\pageref{mathfonts}
را برای دیدن قلم‌های دیگر ریاضی ببینید.

\section{ساختن بلوک‌های فرمولی}

در این بخش، مهمترین فرمان‌های مورد استفاده در حروف‌چینی ریاضی را شرح می‌دهیم. بسیاری از فرمان‌های این بخش احتیاج به 
\textsf{amsmath} ندارند 
(اگر احتیاج داشته باشند، صریحاً بیان می‌شود)
اما به‌هر‌حال این بسته را فراخوانی کنید.


\romanindex{Greek letters}\textbf{\wi{حروف یونانی }کوچک} 
به‌ صورت \verb|\alpha|،  \verb|\beta|، \verb|\gamma|، \ldots، وارد می‌شوند و حروف بزرگ به صورت  \verb|\Gamma|، \verb|\Delta|، \ldots وارد می‌شوند.\footnote{در لاتک حروف بزرگ آلفا، بتا، و غیره تعریف شده نیستند زیرا به شکل \lr{A}، \lr{B}\ldots به نظر می‌رسند. همینکه رمزینه جدید ریاضی تمام شود، همه چیز تغییر می‌کند.}

به جدول  
\ref{greekletters}
%\LRE{\hyperref[greekletters]{2.4}} 
در صفحه 
\pageref{greekletters} برای دیدن لیستی از حروف یونانی نظری بیندازید.
\begin{example}
$\lambda,\xi,\pi,\theta,
 \mu,\Phi,\Omega,\Delta$
\end{example}


\textbf{توان‌ها و اندیس‌ها}
را می‌توان توسط 
\romanindex{exponent}\romanindex{subscript}\index{توان}\index{اندیس} 
\verb|^|
%\index{\verb|^|}
 و 
\verb|_|%\index{\verb|_|}
 نوشت.
بسیاری از فرمان‌ها سبک ریاضی تنها روی اولین حرف بعد از خودشان تأثیر دارند، بنابراین اگر می‌خواهید یک فرمان بر روی چند حرف تأثیر داشته باشد، باید آن حروف را توسط  \verb|{...}| در یک گروه قرار دهید.

جدول  
\ref{binaryrel}
%\LRE{\hyperref[binaryrel]{3.4}} 
در صفحه 
\pageref{binaryrel} شامل بسیاری از عملگر‌ها مانند $\subseteq$ و $\perp$ است.

\begin{example}
$p^3_{ij} \qquad
 m_\text{Knuth} \\[5pt]
 a^x+y \neq a^{x+y}\qquad 
 e^{x^2} \neq {e^x}^2$
\end{example}



\textbf{\wi{رادیکال}} 
توسط \ci{sqrt} و ریشهٔ $-n$ام  به صورت \LRE{\verb|\sqrt[|$n$\verb|]|} نوشته می‌شود. لاتک اندازهٔ علامت رادیکال را به‌طور خودکار مشخص می‌کند. اگر تنها علامت رادیکال مورد نیاز باشد از  \verb|\surd| استفاده کنید.

در جدول  
\ref{tab:arrows}
%\LRE{\hyperref[tab:arrows]{6.4}} 
در صفحهٔ  
\pageref{tab:arrows} دیگر پیکان‌ها مانند  $\hookrightarrow$ و $\rightleftharpoons$ آورده شده‌اند.
\begin{example}
$\sqrt{x} \Leftrightarrow x^{1/2}
 \quad \sqrt[3]{2}
 \quad \sqrt{x^{2} + \sqrt{y}}
 \quad \surd[x^2 + y^2]$
\end{example}


\romanindex{three dots}
\romanindex{vertical dots}
\romanindex{horizontal dots}
\index{سه‌نقطه}
\index{سه‌نقطه!عمودی}
\index{سه‌نقطه!افقی}
معمولاً از نقطه برای نمایش دادن عمل ضرب هنگام کار با نماد‌ها استفاده می‌شود؛ با این وجود گاهی اوقات از چند نقطه برای کمک کردن به خواننده جهت گروه‌بندی فرمول‌ها استفاده می‌شود. برای نوشتن یک نقطه در وسط از \ci{cdot} استفاده می‌شود. \ci{cdots} سه \textbf{\wi{نقطه}}
در وسط قرار می‌دهد درحالی‌که \ci{ldots} نقطه‌ها را روی خط کرسی قرار می‌دهد. بعلاوه،  \ci{vdots} برای قرار دادن عمودی و  \ci{ddots} برای قراردادن کج وجود دارند. مثال‌ دیگری را می‌توانید در بخش 
\ref{sec:arraymat}
%\LRE{\hyperref[sec:arraymat]{2.4.4}}
ببینید.
\begin{example}
$\Psi = v_1 \cdot v_2
 \cdot \ldots \qquad 
 n! = 1 \cdot 2 
 \cdots (n-1) \cdot n$
\end{example}



فرمان‌های \ci{overline} و \ci{underline} \textbf{خط افقی}
درست در بالا یا پایین عبارت قرار می‌دهند: 
\romanindex{horizontal line}
\index{خط!افقی}
\index{خط!عمودی}
\begin{example}
$0.\overline{3} = 
 \underline{\underline{1/3}}$
\end{example}

فرمان‌های \ci{overbrace} و \ci{underbrace}  \textbf{کروشهٔ افقی}
در بالا یا پایین یک عبارت قرار می‌دهند:
\romanindex{horizontal brace} 
\index{براکت!افقی}\index{افقی!براکت} 
\begin{example}
$\underbrace{\overbrace{a+b+c}^6 
 \cdot \overbrace{d+e+f}^9}
 _\text{meaning of life} = 42$
\end{example}


\romanindex{mathematical accents}\index{لهجه!ریاضی} 
برای افزودن لهجه مانند \textbf{پیکان کوچک} 
یا علامت \textbf{\wi{تیلدا}} 
به متغیرها، فرمان‌های ارائه شده در جدول  
\ref{mathacc}
%\LRE{\hyperref[mathacc]{1.4}}
در صفحه 
\pageref{mathacc} ممکن است مفید باشند. 
کلاه و تیلدا که روی چند حرف قرار می‌گیرد با  \ci{widetilde}
و  \ci{widehat} درست می‌شود. به تفاوت بین  محل قرار گرفتن \ci{hat} و \ci{widehat}  \ci{bar} برای متغیرهایی که دارای اندیس هستند توجه کنید.  علامت  
 \verb|'|\Footnote{apostrophe}\romanindex{apostrophe}
%\index{'@\verb"|'"|}
 تولید پرایم 
\index{بیی@پرایم}\romanindex{prime} می‌کند:
% a dash is --
\begin{example}
$f(x) = x^2 \qquad f'(x) 
 = 2x \qquad f''(x) = 2\\[5pt]
 \hat{XY} \quad \widehat{XY}
 \quad \bar{x_0} \quad \bar{x}_0$
\end{example}


\textbf{بردارها}\romanindex{vectors}\index{بردارها} 
اغلب با افزودن یک \wi{علامت پیکان} 
بر روی یک متغیر بدست می‌آیند. این‌کار را با فرمان \ci{vec} انجام می‌دهیم. دو فرمان \ci{overrightarrow} و \ci{overleftarrow} 
برای نشان دادن پیکان از $A$ به $B$ به‌کار می‌روند:
\begin{example}
$\vec{a} \qquad
 \vec{AB} \qquad
 \overrightarrow{AB}$
\end{example}

نام یک تابع مانند لگاریتم اغلب با قلم ایستاده نوشته می‌شود، بنابراین لاتک فرمان‌های زیر را برای نوشتن نام مهمترین توابع به‌کار می‌برد:
\romanindex{mathematical functions}\index{توابع!ریاضی}

\setLR
\begin{tabular}{llllll}
\ci{arccos} &  \ci{cos}  &  \ci{csc} &  \ci{exp} &  \ci{ker}    & \ci{limsup} \\
\ci{arcsin} &  \ci{cosh} &  \ci{deg} &  \ci{gcd} &  \ci{lg}     & \ci{ln}     \\
\ci{arctan} &  \ci{cot}  &  \ci{det} &  \ci{hom} &  \ci{lim}    & \ci{log}    \\
\ci{arg}    &  \ci{coth} &  \ci{dim} &  \ci{inf} &  \ci{liminf} & \ci{max}    \\
\ci{sinh}   & \ci{sup}   &  \ci{tan}  & \ci{tanh}&  \ci{min}    & \ci{Pr}     \\
\ci{sec}    & \ci{sin} \\
\end{tabular}
\setRL

\begin{example}
\[\lim_{x \rightarrow 0}
 \frac{\sin x}{x}=1\]
\end{example}

برای توابعی که در لیست بالا قرار ندارند، از فرمان \ci{DeclareMathOperator}
استفاده کنید. حتی حالت ستاره‌دار این فرمان‌ها برای توابعی که حد بالا یا پایین دارند وجود دارد. این فرمان‌ها تنها در سر‌آغاز باید فعال شوند بنابراین مثال زیر باید در سرآغاز قرار داده شود.
\begin{example}
%\DeclareMathOperator{\argh}{argh}
%\DeclareMathOperator*{\nut}{Nut}
\[3\argh = 2\nut_{x=1}\]
\end{example}
برای \wi{تابع هنگ}\romanindex{modulo function}، دو فرم وجود دارد: \ci{bmod} برای عملگر دوتایی $a \bmod b$ و  \ci{pmod} برای عبارتی به شکل  $x\equiv a \pmod{b}$:
\begin{example}
$a\bmod b \\
 x\equiv a \pmod{b}$
\end{example}

\textbf{کسر}
\index{قیی@کسر}
ایستاده را با فرمان \verb|{...}{...}|\ci{frac} می‌نویسیم. در حالت متنی، کسر کوچک نوشته می‌شود تا در ارتفاع خط قرار بگیرد. این فرم را در سبک نمایشی نیز با  \ci{dfrac} می‌توانید اجرا کنید. اغلب فرم کج 
$1/2$ بهتر است، زیرا برای کسرهای کوچک خواناتر است:

\begin{example}
In display style:
\[3/8 \qquad \frac{3}{8} 
 \qquad \tfrac{3}{8} \]
\end{example}


\begin{example}
In text style:
$1\frac{1}{2}$~hours \qquad
$1\dfrac{1}{2}$~hours
\end{example}




 
در اینجا فرمان \ci{partial} برای \wi{مشتق جزئی}\romanindex{partial derivative} به‌کار رفته است:
\begin{example}
\[\sqrt{\frac{x^2}{k+1}}\qquad
  x^\frac{2}{k+1}\qquad
  \frac{\partial^2f}
  {\partial x^2} \]
\end{example}

برای نوشتن \wi{ضرایب دوجمله‌ای} \romanindex{binomial coefficient} یا چیزهایی شبیه این، از فرمان  \ci{binom} از بستۀ \pai{amsmath} استفاده می‌شود:
\begin{example}
Pascal's rule is
\begin{equation*}
 \binom{n}{k} =\binom{n-1}{k}
 + \binom{n-1}{k-1}
\end{equation*}
\end{example}

برای \wi{عملگرهای دوتایی} \romanindex{binary relations} ممکن است قرار دادن نمادها بر‌روی‌هم مفید باشد. فرمان

\setLR
\ci{stackrel}\verb|{#1}{#2}|
\setRL

\noindent
نماد درون \verb|#1| را به اندازه قلم توان روی \verb|#2| قرار می‌دهد که در محل معمول آن قرار می‌گیرد.
\begin{example}
\begin{equation*}
 f_n(x) \stackrel{*}{\approx} 1
\end{equation*}
\end{example}

\textbf{\wi{عملگر انتگرال}} 
با فرمان \ci{int}, \textbf{\wi{عملگر جمع}} 
با \ci{sum}، و \textbf{\wi{عملگر ضرب}}\romanindex{sum operator}\romanindex{integral operator}\romanindex{product operator} با \ci{prod} تولید می‌شوند. حد بالا و پایین این عملگرها با \verb|^| و \verb|_| مانند اندیس و توان نوشته می‌شوند:
\begin{example}
\begin{equation*}
\sum_{i=1}^n \qquad
\int_0^{\frac{\pi}{2}} \qquad
\prod_\epsilon
\end{equation*}
\end{example}

برای کنترل بیشتر روی محل قرار گرفتن اندیس‌ها در عبارات پیچیده،  \pai{amsmath} فرمان \ci{substack} را ارائه می‌کند:
\begin{example}
\begin{equation*}
\sum^n_{\substack{0<i<n \\ 
        j\subseteq i}}
   P(i,j) = Q(i,j)
\end{equation*}
\end{example}


لاتک همۀ انواع  \textbf{\wi{براکت}} 
و \textbf{\wi{حائل}} \romanindex{braces}\romanindex{delimiters} (مانند $[\;\langle\;\|\;\updownarrow$) را حمایت می‌کند.
براکت‌های گرد و مربعی را می‌توان با کلید مربوط به خودشان نوشت و آکولاد را می‌توان با  \verb|\{| نوشت اما همۀ حائل‌ها را می‌توان با فرمان‌هایی ویژه نوشت (مانند
\verb|\updownarrow|).
\begin{example}
\begin{equation*}
{a,b,c} \neq \{a,b,c\}
\end{equation*}
\end{example}

اگر فرمان \ci{left} را در ابتدای یک حائل چپ، و فرمان \ci{right} را در ابتدای یک حائل راست قرار دهیم، لاتک به‌طور خودکار اندازهٔ حائل را تصحیح می‌کند. توجه داشته باشید که تمام فرمان‌های \ci{left} را باید با فرمان متناظر \ci{right} ببندید. اگر در سمت راست چیزی نمی‌خواهید از  \ci{right} نامرئی استفاده کنید:
\begin{example}
\begin{equation*}
1 + \left(\frac{1}{1-x^{2}}
    \right)^3 \qquad 
\left. \ddagger \frac{~}{~}\right)
\end{equation*}
\end{example}

گاهی اوقات لازم است تا اندازهٔ درست یک حائل ریاضی را دستی تنظیم کنیم \romanindex{mathematical delimiter}\index{حائل!ریاضی}
که با فرمان‌های  \ci{big}، \ci{Big}، \ci{bigg} و 
\ci{Bigg} به عنوان پیشوند بیشتر فرمان‌های حائل امکان‌پذیر است:
\begin{example}
$\Big((x+1)(x-1)\Big)^{2}$\\
$\big( \Big( \bigg( \Bigg( \quad
\big\} \Big\} \bigg\} \Bigg\} \quad
\big\| \Big\| \bigg\| \Bigg\| \quad
\big\Downarrow \Big\Downarrow 
\bigg\Downarrow \Bigg\Downarrow$
\end{example}
 برای دیدن لیست کاملی از حائل‌ها جدول  
\ref{tab:delimiters}
%\LRE{\hyperref[tab:delimiters]{8.4}}
در صفحه 
\pageref{tab:delimiters} را ببینید. 
\section{تنظیم عمودی}
\subsection{فرمول‌های چندگانه}
\romanindex{multiple equation}\index{فرمول‌!چندگانه}

برای فرمول‌هایی که در چند خط قرار می‌گیرند یا برای \wi{دستگاه معادلات}
 \romanindex{equation system},
می‌توانید از محیط \ei{align} و \verb|align*| به جای  \texttt{equation} و \texttt{equation*} استفاده کنید.\footnote{محیط  \ei{align} از بستۀ  \textsf{amsmath} است. محیط مشابه به  این محیط در خود لاتک با عنوان \ei{eqnarray} وجود دارد، اما عموماً توصیه نمی‌شود زیرا مکان و برچسب آن پایدار نیست.} 
با \ei{align} هر خط معادله یک شماره می‌گیرد. \verb|align*| هیچ چیز را شماره‌گذاری نمی‌کند.

محیط \ei{align} یک معادله را  پیرامون علامت \verb|&| گرد می‌کند.
فرمان \verb|\\| خط‌ها را می‌شکند. اگر می‌خواهید یک معادله را شماره‌گذاری نکنید از فرمان \ci{nonumber} برای حذف شمارهٔ آن استفاده کنید. این فرمان باید \femph{قبل}
از \verb|\\| قرار داده شود:
\begin{example}
\begin{align}
f(x) &= (a+b)(a-b) \label{1}\\
     &= a^2-ab+ba-b^2  \\ 
     &= a^2+b^2 \tag{wrong}
\end{align}
This is a reference to \eqref{1}.
\end{example}

\index{فرمول‌های طولانی}\romanindex{long equations} 
\textbf{فرمول‌های طولانی} 
به صورت خودکار  شکسته نمی‌شوند. نویسنده باید مشخص کند کجا باید شکسته شوند و تورفتگی مناسب را مشخص کند:
\begin{example}
\begin{align}
f(x) &= 3x^5 + x^4 + 2x^3 
                \nonumber \\
     &\qquad + 9x^2 + 12x + 23 \\
     &= g(x) - h(x)
\end{align}
\end{example}
بستۀ \pai{amsmath} چند محیط مفید دیگر را نیز در بر دارد: \verb|flalign|،
\verb|gather|، \verb|multline| و \verb|split|. برای اطلاعات بیشتر به راهنمای این بسته مراجعه کنید.
\subsection{آرایه و ماتریس} \label{sec:arraymat}

برای حروف‌چینی آرایه‌ها از محیط \ei{array} استفاده کنید. این محیط شبیه محیط  \texttt{tabular} است. فرمان \verb|\\| برای شکستن خط‌ها به‌کار می‌رود:
\begin{example}
\begin{equation*}
 \mathbf{X} = \left( 
  \begin{array}{ccc}
   x_1 & x_2 & \ldots \\
   x_3 & x_4 & \ldots \\
   \vdots & \vdots & \ddots
  \end{array} \right)
\end{equation*}
\end{example}


از محیط \ei{array} همچنین برای نوشتن 
\romanindex{piecewise function}\wi{توابع چند‌ضابطه} 
توسط یک \verb|.| به عنوان یک حائل راست نامرئی استفاده می‌شود:\footnote{اگر می‌خواهید خیلی از این فرم استفاده کنید محیط \ei{cases} از بستۀ 
  \textsf{amsmath} کار را بسیار راحت می‌کند و بنابراین ارزش نگاه کردن را دارد.}  
\begin{example}
\begin{equation*}
|x| = \left\{
 \begin{array}{rl}
  -x & \text{if } x < 0\\
   0 & \text{if } x = 0\\
   x & \text{if } x > 0
 \end{array} \right.
\end{equation*}
\end{example}



\ei{array} را می‌توان برای نوشتن ماتریس‌ها \index{ماتریس}\romanindex{matrix} نیز به‌کار برد، اما 
\pai{amsmath} راه‌ حل بهتری را توسط محیط \ei{matrix} پیشنهاد می‌کند. شش نسخه از آن با حائل‌های مختلف وجود دارد: \ei{matrix}
(خالی)، \ei{pmatrix} $($، \ei{bmatrix} $[$، \ei{Bmatrix} $\{$، \ei{vmatrix} $\vert$ و
\ei{Vmatrix} $\Vert$. با \ei{array} لازم نیست تعداد ستون‌ها را مشخص کنید. بیشترین تعداد ستون ۱۰ 
است اما قابل تغییر است 
(هرچند معمولاً بیشتر از ۱۰ ستون لازم نیست!).
\begin{example}
\begin{equation*}
 \begin{matrix} 
   1 & 2 \\
   3 & 4 
 \end{matrix} \qquad
 \begin{bmatrix} 
   1 & 2 & 3 \\
   4 & 5 & 6 \\ 
   7 & 8 & 9
 \end{bmatrix}
\end{equation*}
\end{example}


\section{فاصله در محیط ریاضی}\label{sec:math-spacing}

\romanindex{math spacing}\index{فاصلهٔ ریاضی} 
اگر فاصلهٔ انتخاب شده توسط لاتک در فرمول‌ها مناسب نیست، می‌توان آن را با فرمان‌هایی تصحیح کرد: \ci{,} برای
$\frac{3}{18}\:\textrm{quad}$ (\demowidth{0.166em})، \ci{:} برای $\frac{4}{18}\:
\textrm{quad}$ (\demowidth{0.222em}) و  \ci{;} برای $\frac{5}{18}\:
\textrm{quad}$ (\demowidth{0.277em}).  حرف فرار \verb*|\ | تولید یک فاصله بین  \ci{quad}
(\demowidth{1em}) و \ci{qquad} (\demowidth{2em}) می‌کند. اندازهٔ 
\ci{quad} متناظر با عرض حرف \lr{`M'} از این قلم جاری است.  \verb|\!|
%\cih{"!} 
تولید یک فاصلهٔ منفی به اندازهٔ  $-\frac{3}{18}\:\textrm{quad}$ ($-$\demowidth{0.166em}) می‌کند.

توجه کنید \lr{`d'} در عملیات دیفرانسیل به خوبی در قلم ایستاده نوشته می‌شود:
\begin{example}
\begin{equation*}
 \int_1^2 \ln x \mathrm{d}x \qquad
 \int_1^2 \ln x \,\mathrm{d}x
\end{equation*}
\end{example}


در مثال بعد، تابع جدید \ci{ud} را تعریف می‌کنیم که نماد $\,\mathrm{d}$ را تولید می‌کند (به فاصلهٔ  
\demowidth{0.166em}
قبل از 
$\text{d}$ توجه داشته باشید)،
بنابراین لازم نیست هربار آن را بنویسیم. فرمان  \ci{newcommand} در سرآغاز آورده می‌شود.
\begin{example}
\newcommand{\ud}{\,\mathrm{d}}

\begin{equation*}
 \int_a^b f(x)\ud x 
\end{equation*}
\end{example}

اگر می‌خواهید انتگرال چندگانه را بنویسید، خواهید دید که فاصله بین انتگرال‌ها نامطبوع است. می‌تواید این فاصله را با فرمان  \ci{!}
%\cih{"!}
 تغییر دهید، اما بستۀ 
\pai{amsmath}
 راه حل ساده‌تری برای این‌کار دارد که عبارت است از  
\ci{iint}، \ci{iiint}، \ci{iiiint}،  و \ci{idotsint}.

\begin{example}
\newcommand{\ud}{\,\mathrm{d}}

\[ \int\int f(x)g(y) 
                  \ud x \ud y \]
\[ \int\!\!\!\int 
         f(x)g(y) \ud x \ud y \]
\[ \iint f(x)g(y) \ud x \ud y \]
\end{example}

برای اطلاعات بیشتر به راهنمای الکترونیکی  \texttt{testmath.tex} از \lr{\AmS-\LaTeX} یا فصل ۸ از \companion{} مراجعه کنید.
\subsection{اشباح}

وقتی فرمول‌های مرتب عمودی شامل  \verb|^| و  \verb|_| می‌نویسید، گاهی اوقات لاتک خیلی کمک نمی‌کند. با استفاده از فرمان  \ci{phantom} می‌توانید فضایی برای حرفی که نمی‌خواهید در خروجی ظاهر شود ایجاد کنید. راحت‌ترین راه برای فهمیدن این موضوع مثال زیر است:
\begin{example}
\begin{equation*}
{}^{14}_{6}\text{C}
\qquad \text{versus} \qquad
{}^{14}_{\phantom{1}6}\text{C}
\end{equation*}
\end{example}
اگر می‌خواهید تعداد زیادی از ایزو‌توپ‌ها را همانند مثال بالا بنویسید، بستۀ  \pai{mhchem} برای نوشتن فرمول‌های شیمی بسیار مفید است.
\section{ریزه‌کاری با قلم‌های ریاضی}\label{sec:fontsz}
قلم‌های مختلف ریاضی را در جدول  
\ref{mathalpha}
%\LRE{\hyperref[mathalpha]{14.4}} 
در صفحه 
\pageref{mathalpha} آورده‌ایم.
\begin{example}
 $\Re \qquad
  \mathcal{R} \qquad
  \mathfrak{R} \qquad
  \mathbb{R} \qquad $  
\end{example}
دوتای آخر به  \pai{amssymb} یا  \pai{amsfonts} احتیاج دارند.

گاهی اوقات باید به لاتک بگویید که اندازه را تصحیح کند. در سبک ریاضی، این‌کار را با فرمان زیر انجام می‌دهیم:

\begin{latin}
\ci{displaystyle}~($\displaystyle 123$),
 \ci{textstyle}~($\textstyle 123$), 
\ci{scriptstyle}~($\scriptstyle 123$) \rl{و}\\
\ci{scriptscriptstyle}~($\scriptscriptstyle 123$).
\end{latin}

اگر $\sum$ در یک کسر قرار داشته باشد، به سبک متنی حروف‌چینی می‌شود مگر این که به لاتک اطلاع دهید:
\begin{example}
\begin{equation*}
 P = \frac{\displaystyle{ 
   \sum_{i=1}^n (x_i- x)
   (y_i- y)}} 
   {\displaystyle{\left[
   \sum_{i=1}^n(x_i-x)^2
   \sum_{i=1}^n(y_i- y)^2
   \right]^{1/2}}}
\end{equation*}    
\end{example}
تغییر سبک عموماً روی عملگرهای بزرگ و حدود آنها تاثیر می‌گذارد.

% This is not a math accent, and no maths book would be set this way.
% mathop gets the spacing right.

\subsection{حروف سیاه}
\romanindex{bold symbols}\index{حروف سیاه}

نوشتن حروف سیاه در لاتک سخت است؛ یک حروف‌چین‌ آماتور ممکن است بخواهد بیش‌ از ‌حد از حروف سیاه استفاده کند. فرمان تغییر قلم \verb|\mathbf| حروف سیاه را تولید می‌کند، اما این حروف ایستاده هستند و نمادهای ریاضی ایتالیک هستند، و یک فرمان  \ci{boldmath} وجود دارد، \femph{این فرمان تنها باید در خارج از سبک ریاضی مورد استفاده قرار گیرد}. 
با این وجود از آن می‌توان برای نماد‌ها نیز استفاده کرد:
\begin{example}
$\mu, M \qquad 
\mathbf{\mu}, \mathbf{M}$
\qquad \boldmath{$\mu, M$}
\end{example}

بستۀ  \pai{amsbsy} (توسط \pai{amsmath} توزیع می‌شود) 
و همچنین 
\pai{bm} از کلاف \texttt{tools} این‌کار را با ارائه فرمان  \ci{boldsymbol} راحت‌تر می‌کنند:

\begin{example}
$\mu, M \qquad
\boldsymbol{\mu}, \boldsymbol{M}$
\end{example}
\section{قضیه‌ها، قانون‌ها}

هنگام نوشتن نوشتار‌‌ ریاضی، ممکن است به نوشتن ساختار‌هایی مانند قضیه، تعریف، اصل، و غیره احتیاج پیدا کنید.
\begin{lscommand}
\ci{newtheorem}\verb|{|\emph{name}\verb|}[|\emph{counter}\verb|]{|%
         \emph{text}\verb|}[|\emph{section}\verb|]|
\end{lscommand}
آرگومان \emph{name} کلمه کلیدی برای شناسایی \lr{theorem} است. با آرگومان \emph{text} نام واقعی قضیه را معرفی می‌کنید که در خروجی چاپ می‌شود.

آرگومان‌های درون کروشه اختیاری هستند. از آنها برای مشخص کردن نوع شماره‌گذاری قضیه استفاده می‌شود. از آرگومان \emph{counter} 
برای همنوع شدن شماره‌گذاری با یک شماره‌گذاری تعریف شده استفاده می‌شود. آرگومان \emph{section} اجازه می‌دهد در شماره قضیه شماره بخش نیز وارد شود.

بعداز اجرای فرمان \ci{newtheorem} در سرآغاز مستندتان، می‌توانید از محیط تعریف شده در نوشتار‌ به شکل زیر استفاده کنید.
\begin{code}
\verb|\begin{|\emph{name}\verb|}[|\emph{text}\verb|]|\\
\lr{This is my interesting theorem}\\
\verb|\end{|\emph{name}\verb|}|     
\end{code}

بستۀ \pai{amsthm} (قسمتی از \lr{\AmS-\LaTeX}) 
فرمان \verb|}|\emph{style}\verb|{|\ci{theoremstyle}
را ارائه می‌کند که توسط آن می‌توانید از محیط‌های از پیش‌تعریف‌شده  مانند \texttt{definition} (تیتر بزرگ، بدنه رومن)،
\texttt{plain} (تیتر بزرگ، بدنه ایتالیک) 
یا \texttt{remark} (تیتر ایتالیک، بدنه رومن) 
استفاده کنید.

تئوری بس است. مثال‌های زیر هر نوع ابهامی را از بین می‌برد و مشخص می‌کند محیط 
\verb|\newtheorem| کمی برای فهمیدن مشکل است.

% actually define things
\begin{latin}
\theoremstyle{definition} \newtheorem{law}{Law}
\theoremstyle{plain}      \newtheorem{jury}[law]{Jury}
\theoremstyle{remark}     \newtheorem*{marg}{Margaret}
\end{latin}

ابتدا قضیه‌ها را تعریف می‌کنیم:
\begin{latin}
\begin{verbatim}
\theoremstyle{definition} \newtheorem{law}{Law}
\theoremstyle{plain}      \newtheorem{jury}[law]{Jury}
\theoremstyle{remark}     \newtheorem*{marg}{Margaret}
\end{verbatim}
\end{latin}


\begin{example}
\begin{law} \label{law:box}
Don't hide in the witness box
\end{law}
\begin{jury}[The Twelve]
It could be you! So beware and
see law~\ref{law:box}.\end{jury}
\begin{marg}No, No, No\end{marg}
\end{example}


قضیهٔ \lr{Jury} دارای شماره‌گذاری \lr{Law} است، بنابراین شماره‌ای را اخذ می‌کند که در دنبالهٔ شمارهٔ \lr{Laws} است.  آرگومان داخل کروشه برای معین کردن یک عنوان شبیه قضیه است.


\begin{example}
\newtheorem{mur}{Murphy}[section]


\begin{mur} If there are two or 
more ways to do something, and 
one of those ways can result in
a catastrophe, then someone 
will do it.\end{mur}
\end{example}



قضیهٔ \lr{Murphy} شماره‌ای وابسته به شمارهٔ بخش جاری اخذ می‌کند. می‌توانید به‌جای بخش از فصل و شبیه آن استفاده کنید.

بستۀ \pai{amsthm} دارای محیط \ei{proof} نیز است.

\renewcommand\proofname{Proof}
\begin{example}
\begin{proof}
 Trivial, use
\[E=mc^2\]
\end{proof}
\end{example}


با فرمان \ci{qedhere} می‌توانید علامت انتهای اثبات را در مواقعی که به‌تنهایی در یک خط قرار دارد در مکان مناسبی درج کنید.

\begin{example}
\begin{proof}
 Trivial, use
\[E=mc^2 \qedhere\]
\end{proof}
\end{example}

اگر می‌خواهید تا محیط مناسبی برای خود طراحی کنید، بستۀ \pai{ntheorem} گزینه‌های بسیار زیادی در اختیارتان قرار می‌دهد.



% Local Variables:
% TeX-master: "lshort"
% mode: latex
% mode: flyspell
% End:
