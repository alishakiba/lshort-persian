\chapter{پیشگفتار مترجم}
امروزه اکثر مجله‌های علمی و پژوهشی از نویسندگان خود انتظار دارند که مقالهٔ خود را با لاتک تهیه کنند. مشهور است که کسانی که اولین بار با لاتک متنی را آماده می‌کنند، در میانهٔ کار می‌گویند که دیگر از این نرم‌افزار استفاده نخواهند کرد؛ اما بعد از اتمام کار به خود می‌گویند دیگر به هیچ عنوان به سراغ نرم‌افزارهایی مانند \lr{word} نخواهند رفت. دلیل این کار واضح است؛ لاتک برای هر منظور، فرمانی دارد که باید این فرمان‌ها را بدانید تا بتوانید به طور بهینه از آن استفاده کنید. اگر در ابتدا زمان کافی برای یادگیری این فرمان‌ها صرف نکنید، در آینده چندین برابر این زمان را برای رفع مشکلات نوشتار خود باید صرف کنید. این کتاب به این امید ترجمه شده است که بتواند به شما در یادگیری فرمان‌های لاتک کمک کند.

به تازگی نرم‌افزار زیتک به بازار ارائه شده است که توانایی استفاده از قلم‌های مختلف را فراهم کرده است. زیلاتک، که همان لاتک بر پایهٔ زیتک است، تمامی امکانات قوی لاتک را برای تهیهٔ هر نوع مستندی، از جمله مستندات فارسی، ارائه کرده است. به همین منظور بسته‌ای با نام \lr{\XePersian} توسط آقای وفا خلیقی تهیه شده است که این ترجمه با استفاده از این بسته و به منظور بررسی سازگاری آن تهیه شده است. آقای وفا خلیقی دانشجوی دکتری ریاضی دانشگاه سیدنی هستند که واقعاً با تلاش غیرقابل توصیف کار تهیهٔ این بسته را به عهده گرفتند و بدون چشم‌داشتی این کار بزرگ را انجام دادند. وظیفهٔ خود می‌دانم که از طرف جامعهٔ علمی کشور از ایشان کمال تشکر را داشته باشم و با افتخار این ترجمهٔ ناچیز را به خود ایشان تقدیم کنم. 

همچنین لازم است از زحمات آقای مصطفی واحدی به خاطر شروع اولین قدم‌های تهیهٔ بسته‌ای برای نگارش فارسی و همچنین مبدل فارسی‌تک به یونیکد (به سبک مناسب زی‌پرشین) و همچنین ایجاد گروه فارسی لاتک گوگل%
\Footnote{\href{http://groups.google.com/group/farsilatex?hl=fa}{\texttt{http://groups.google.com/group/farsilatex?hl=fa}}}
تشکر نمایم.  برای دریافت کمک و انتقال نظرات و پیشنهادات خود و همچنین دریافت آخرین اطلاعات می‌توانید به این گروه ملحق شوید. امکانات استفاده از \texttt{BibTex} توسط آقای محمود امین طوسی فراهم گردیده است که از ایشان سپاسگذاری می‌کنم. از آقای سید رضی علوی‌زاده برای تهیهٔ افزونهٔ نگارش فارسی به ویرایشگر \lr{Texmaker} و از آقای امیرمسعود پورموسی برای تلاش بسیار ایشان در آماده‌سازی ویکی زی‌پرشین%
\Footnote{\href{http://fa.parsilatex.wikia.com}{\texttt{http://fa.parsilatex.wikia.com}}}
تشکر می‌کنم.
\begin{latin}
\contrib{\rl{مهدی امیدعلی}}{mehdioa@gmail.com}{}
\end{latin}