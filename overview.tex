%%%%%%%%%%%%%%%%%%%%%%%%%%%%%%%%%%%%%%%%%%%%%%%%%%%%%%%%%%%%%%%%%
% Contents: Who contributed to this Document
% $Id: overview.tex 169 2008-09-24 07:32:13Z oetiker $
%%%%%%%%%%%%%%%%%%%%%%%%%%%%%%%%%%%%%%%%%%%%%%%%%%%%%%%%%%%%%%%%%

% Because this introduction is the reader's first impression, I have
% edited very heavily to try to clarify and economize the language.
% I hope you do not mind! I always try to ask "is this word needed?"
% in my own writing but I don't want to impose my style on you... 
% but here I think it may be more important than the rest of the book.
% --baron

\chapter{پیشگفتار}

\lr{\LaTeX{}} \cite{manual}
یک سیستم حروف‌چینی است که برای تولید نوشتار‌‌ با کیفیت عالی علمی و ریاضی بسیار مناسب است. این سیستم همچنین برای تولید انواع دیگر نوشتار‌‌، از یک نامهٔ ساده تا کتاب‌های کامل، مناسب است. 
\lr{\LaTeX}
از 
\lr{\TeX} \cite{texbook}
به عنوان موتور حروف‌چین استفاده می‌کند.

این مقدمهٔ کوتاه به معرفی لاتک می‌پردازد و برای بسیاری از کاربردهای آن کافی است. برای مشاهدهٔ شرح کاملی از سیستم لاتک به 
\cite{manual,companion}
مراجعه کنید.

\bigskip
\noindent 
این مقدمه به ۶ فصل تقسیم می‌شود:
\begin{description}
\item[فصل ۱] 
شما را از ساختار ابتدایی نوشتارهای لاتک آگاه می‌سازد. همچنین کمی از تاریخچهٔ لاتک نیز در این فصل گنجانده شده است. بعد از مطالعهٔ این فصل، شمایی کلی از روش کار لاتک را می‌آموزید.
\item[فصل ۲] 
به درون جزئیات حروف‌چینی نوشتار سفر می‌کند. این فصل بیشتر فرمان‌ها و محیط‌های اساسی لاتک را معرفی و تشریح می‌کند. بعد از مطالعهٔ این فصل، توانایی تولید نوشتار خود را خواهید داشت.
\item[فصل ۳]
روش نگارش فرمول‌ها را در لاتک شرح می‌دهد. مثال‌های زیادی برای توضیح کامل قدرت واقعی لاتک در این زمینه ارائه شده است. در انتهای این فصل تمام نمادهای موجود لاتک  در چندین جدول آورده شده است.
\item[فصل ۴] 
روش تولید نمایه و کتاب‌نامه، و الصاق تصویر‌های ای.پی.اس را شرح می‌دهد. همچنین روش تولید نوشته‌های پی.دی.اف به وسیلهٔ پی.دی.اف.لاتک بیان می‌شود و چندین بستهٔ گسترش‌یافته معرفی می‌شود. 
\item[فصل ۵] 
روش تولید شکل‌ را با کمک لاتک شرح می‌دهد. به جای رسم شکل‌ها به وسیلهٔ برنامه‌های کامپیوتری، ذخیره و الصاق آنها، یاد می‌گیرید که این شکل‌ها را چگونه در لاتک معرفی کنید و آنگاه لاتک آنها را برای شما رسم می‌کند.
\item[فصل ۶] 
شامل اطلاعاتی خطرناک برای تغییر طرح نوشتار در لاتک است. این فصل به شما یاد می‌دهد که، بسته به توانایی شما، چگونه چیز‌هایی را تغییر دهید تا طرح زیبای خروجی لاتک را به شکلی زشت و ناراحت‌کننده تبدیل کنید.
\end{description}
\bigskip
\noindent 
بسیار مهم است که فصل‌های این مقدمه را به ترتیب مطالعه کنید --- این کتاب آنقدر پرحجم نیست. مطمئن شوید که تمام مثال‌ها را به دقت مطالعه کرده‌اید، زیرا حجم گسترده‌ای از اطلاعات این کتاب در مثال‌هایش نهفته است.

\bigskip
\noindent 
لاتک برای بسیاری از انواع کامپیوترها، از کامپیوترهای شخصی گرفته تا مکینتاش و سیستم‌های بزرگ یونیکس و وی.ام.اس، وجود دارد. بر روی بسیاری از کامپیوترهای دانشگاه‌ها این سیستم نصب و آمادهٔ استفاده است. نصب خانگی لاتک در 
\guide
شرح داده شده است. اگر در نصب این سیستم به مشکل برخوردید، از کسی که این کتاب را به شما داده است کمک بگیرید. هدف این کتاب راهنمایی شما برای نصب لاتک نیست، بلکه هدف آن راهنمایی برای تولید نوشتار توسط لاتک است.

\bigskip
\noindent 
اگر به چیزهایی وابسته به لاتک احتیاج دارید، نگاهی به وبگاه شبکه آرشیو بزرگ تک 
(\lr{CTAN})
بیندازید. صفحهٔ خانگی این آرشیو در 
\lr{\texttt{http://www.ctan.org}}
قرار دارد. 
همهٔ بسته‌های لاتک را می‌توانید از آرشیو اف.تی.پی 
\lr{\texttt{ftp://www.ctan.org}}
و سایت‌های آینه‌ای آن در سراسر جهان دریافت کنید.

در کتاب ارجاع‌های دیگری به 
\lr{\texttt{CTAN}}
خواهید یافت، که به طور ویژه به نوشته‌ها و نرم‌افزارهایی مورد نیاز اشاره می‌کنند. به جای نوشتن متن کامل 
\lr{url}،
تنها کلمهٔ 
\lr{\texttt{CTAN}}
را به همراه شاخه‌ای که باید بروید، نوشته‌ام.

اگر می‌خواهید لاتک را روی کامپیوتر خود راه‌اندازی کنید، به آدرس زیر نگاهی بیندازید:

\setLR{\CTAN|systems|}\setRL


\vspace{\stretch{1}}
\noindent 
اگر نظری برای اضافه یا کم کردن این مقدمه دارید، لطفاً مرا مطلع سازید.  در این رابطه که چه قسمت از این مقدمه مناسب است و چه قسمت باید بیشتر توضیح داده شود، بسیار مایل هستم که دیدگاه‌های افراد تازه‌کار رابدانم.

\bigskip

\begin{latin}
\begin{verse}
\contrib{Tobias Oetiker}{tobi@oetiker.ch}%
\noindent{OETIKER+PARTNER AG\\Aarweg 15\\4600 Olten\\Switzerland}
\end{verse}
\vspace{\stretch{1}}
\end{latin}


%

% Local Variables:
% TeX-master: "lshort2e"
% mode: latex
% mode: flyspell
% End:
